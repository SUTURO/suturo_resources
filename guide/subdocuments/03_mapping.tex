\documentclass[main.tex]{subfiles}
\begin{document}
	\chapter{Mapping}
	
	\section{Requirements}
	\begin{itemize}
		\item Controller
		\item pkg: mapping\_hector\_slam
		\item pkg: map\_server
	\end{itemize}
	
	\section{Setup}
	\begin{itemize}
		\item Open RVIZ
		\subitem Make sure laser scans are received
		\subitem (optional) Set fixed\_frame to map
		\item Clear the area off all Objects and or humans that should not be part of the map.
		\item If necessary adjust the parameters for hector\_mapping in the launchfile 
		\subitem map\_size: should be adjusted according to the dimension of the room.
		\subitem resolution: should be adjusted according to your needs. 
	\end{itemize}
	
	\section{Creating the map}
	\begin{itemize}
		\item rosnode kill /pose\_integrator
		\item roslaunch mapping\_hector\_slam hector\_slam.launch
		\item Drive around the robot
	\end{itemize}
	
	\section{Saving the map}
	\begin{itemize}
		\item rosrun map\_server map\_saver -f path\_to\_map
		\item Open the created .yaml file and adjust the path to the .pgm file of the map.
	\end{itemize}
	
	\section{Notes}
	\begin{itemize}
		\item The position and rotation of the robot is crucial for the root position and rotation of the map so choose it wisely.
		\item While mapping the area try drive in a route, that always allows the laser scanner to keep track of two walls or other big obstacles.
		\item Another good idea is to first do a round of the outer perimeter of the area along the walls and afterwards the area inside. This helps improving the accuracy.
		\item If the first map does not look good just restart the hector\_slam.launch file.
	\end{itemize}
	
	

\end{document}