\documentclass[main.tex]{subfiles}
\begin{document}
	
	\chapter{Challenges}
	\chapterauthor{Fabian Rosenstock}

\section{General}

While working on this milestones goals, we had to completely change our way of working, due to the Corona virus outbreak.
We could not work with the robot anymore and the RoboCup got canceled. This lead to a loss of motivation, since we lost our main goal and could not personally work together anymore.
The situation forced us to work in a simulation. We had to set up a fitting simulation environment and find a way to properly integrate and test our system from home. This did cost a lot of time, where we were not able to properly work on the project.
These changes had some positive side effects. Since we worked from home, it was easier to contact someone who was able and willing to help when we experienced any difficulties with a component. Most of the time we worked on one computer, which lead to an better overview over the work and the problems of other groups.

\section{Planning}

The Corona changes made the work for planning really hard, since they worked close with the robot. Because of that they could not really work while the changes were made and had to adapt to the new situation, which took some time.
This lead to a loss in motivation. Despite all that the group functioned well, with solid communication and clear division of tasks.

\section{Perception}

The teamwork and communication went well, despite the Corona virus. It even got too simple at times due to perfect conditions in the simulator, Perception did not face the real world problems of bad quality pictures or other recognition problems. Most parts of the challenge in working with different lights etc. were gone.

\section{Knowledge}

A clear division of tasks lead to good teamwork and easier communication in the knowledge group.
The changes made due to the Corona virus outbreak, did not affect the knowledge group as much as the other groups, since they did not have to work as closely with the robot.

\section{Manipulation}

The manipulation group worked a lot with the robot, which went well, since the communication and teamwork was good.
Because of that they were strongly affected by the changes made due to the Corona virus. They had a hard time to enable the robot to do the same things in the simulation as in the real world.

\section{Navigation}
At the end of the last milestone it became apparent, that the HSR hat some serious problems moving around. The origin of this problem most likely lies in the actual hardware. Due to Corona access to the HSR and further investigations became impossible. The navigation inside of the simulation worked flawlessly. Instead the focus was shifted to the object finder. A small node trying to find objects on the ground with the laser scanner. The program still had some problems detecting the position of an object correctly while the HSR moved around. Additionally it had been tweaked to work with the original HSR, which made an adjustment of its parameters necessary.  

\end{document}
