\documentclass[main.tex]{subfiles}
\begin{document}
	
	\chapter{Perception}
	\chapterauthor{Leonidas Paniago De Oliveira Neto}
	
The main focus for perception in this milestone was on fine tuning and development of non essential features, with the Shape Annotator and classification for simulated objects as the most relevant changes.
Door detection is implemented but not used or tested due to the covid 19 circumstances.
The tracking of objects with 2D images is possible shown in the python prototype, but not realized due to version conflicts in ROS.
As a alternative to 3D classification there is now a 2D classifier using the GoogleLeNet neural network and caffe as training framework.
The classifier evaluation is functional but due to time constraints is the user interface not implemented.
In conclusion perception did fulfill all necessary adjustments for the migration in to the simulated environment and still delivered useful improvements.

	
\end{document}
