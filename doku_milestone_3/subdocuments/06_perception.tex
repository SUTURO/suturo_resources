\documentclass[main.tex]{subfiles}
\begin{document}
	
	\chapter{Perception}
	\chapterauthor{Leonidas Paniago De Oliveira Neto}
	
The main focus for perception in this milestone was on fine tuning and development of non essential features, with the Shape Annotator and classification for simulated objects as the most relevant changes.

Door detection is implemented but not used or tested due to the Covid-19 circumstances.

The tracking of objects with 2D images is possible, which is shown in the python prototype, but not realized due to version conflicts in ROS.
As an alternative to 3D classification there is now a 2D classifier using the GoogleLeNet neural network and caffe as training framework.

Classification evaluation is functional but due to time constraints the user interface has not been implemented. See chapter 7.7.1 in the final project report for details on how to use the package.

In conclusion perception did fulfill all necessary adjustments for the migration in to the simulated environment and still delivered useful improvements.

	
\end{document}
