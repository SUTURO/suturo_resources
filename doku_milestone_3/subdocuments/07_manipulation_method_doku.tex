\documentclass[main.tex]{subfiles}

\begin{document}

	\begingroup

	\renewcommand{\cleardoublepage}{}

	\renewcommand{\clearpage}{}

	\chapter{Manipulation Function Documentation}


		\chapterauthor{\vspace{5mm}Marc Stelter, Jan Neuman \& Fabian Weihe}
		
		\subsection{General}

		Manipulation has four Actionservers.\\
		The "move\_gripper\_action\_server" moves the gripper to the given position.\\
		The "take\_pose\_action\_server" moves the robot into perceiving position.\\
		The "grasps\_object\_server" receives the size and position of an object and grasps it.\\
		The "place\_server" places the attached object at the given pose.\\
		Further Manipulation has a "suturo\_manipulation\_launch" package, which launches the Actionservers, rviz and giskard and a "object\_state\_listener" package.

		\subsection{Grasp Object}
			The script recives a string ("object\_frame\_id"), a PoseStamped ("goal\_pose"), a Vector3 ("object\_size") and an uint8 (grasp\_mode).
			
			\vspace{0,25 cm}
			The "object\_frame\_id" relates to the object we want to grasp.
		
			\vspace{0,25 cm}
			The "goal\_pose" is the position of the object in the map.

			\vspace{0,25 cm}
			"object\_size" contains the size of the object, we want to grasp.
			
			\vspace{0,25 cm}
			The "grasp\_mode" says in which way we want to grasp the object. \\ 
			0 ("FREE"): we use the given orientation.\\
			 1 ("FRONT"): grasp the object frontal.\\
			 2 ("TOP"): grasp the object from the top\\
			 3 ("JUST"): just close the gripper with trasport pose.
			 
			 \vspace{0,75 cm}
			 The first method "get\_current\_joint\_state"  sets the value of every slider to ists corresponding current joint state.
			 
			 \vspace{0,25 cm}
			 "execute\_cb" is the first method wich is called, when the server recives a new goal. Based on wich "grasp\_mode" is choosen the method either call "grasp\_to\_position" or "just\_grasp". The only case where "just\_grasp" is called is in "grasp\_mode" 3 ("JUST").
			 
			 \vspace{0,25 cm}
			 In every other case "grasp\_to\_position" is called.\\
			In the first step the method opens the gripper to the maximum. After that the arm is moved to the object. The "grasp\_mode" determines if the object should be grasped from the front, the top or a free given orientation. When the arm has moved to the given position "goal\_pose", the script closes the gripper. At the end the grasped object given by "object\_frame\_id" is added in giskard and attached to the robot. After that the method "move\_to\_transport\_pose" is called.\\
			Finally it checks if the object is in the gripper and returns whether the action has failed or succeeded.
			
			\vspace{0,25 cm}
			The method "just\_grasp" move to the transport pose, opens the griper, wait 3 seconds and close the gripper. Finally it checks if the object is in the gripper.
			
			\vspace{0,25 cm}
			The method "move\_to\_transport\_pose" transform the robot in a position, where it can comfortably move arround.
			
			\vspace{0,25 cm}
			The last method "object\_in\_gripper" checks if the robot hold something in the gripper and return a boolean.
			 


			
	\endgroup

\end{document}
