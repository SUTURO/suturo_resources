\documentclass[main.tex]{subfiles}

\begin{document}
	
	\begingroup
	
	\renewcommand{\cleardoublepage}{}
	
	\renewcommand{\clearpage}{}
	
	\chapter{Grocery Task Overview}
	
	
	\chapterauthor{}
	
	\section{Goal}

	The goal which has to be achieved in the grocery task is to collect objects from a table and place them in a logical position in a shelf. The HSR has to autonomusly navigate the environment, it has to manipulate objects like doors and grocery items like chips, bananas etc. furthermore the HSR has to be able to lift items of the table and place them in the shelf where the final position of the moved object should either be grouped with similar objects (similarity is defined by size, color, shape and class) or open a new group apart from items which are not similar. The HSR has 5 minutes to achieve the given task. HIER FEHLT NOCH GENAUERE BESCHREIBUNG
	

	\section{Tasks}

		\begin{figure}	
			\centering
			\includegraphics[width=0.85\textwidth]{pictures/diagramms/first-part-grocery-sequence.png}
			\caption{Sequence diagram of the complete run of the grocery storing task \textit{(explanations below)}}
			\label{grocery_seq_01}
		\end{figure}
		\begin{figure}	
			\centering
			\includegraphics[width=0.85\textwidth]{pictures/diagramms/second-part-grocery-sequence.png}
			\caption{Sequence diagram of the complete run of the grocery storing task \textit{(explanations below)}}
			\label{grocery_seq_02}
		\end{figure}
	
	\subsection{Setup}
		The sequence diagramm in figure \ref{grocery_seq_01} and \ref{grocery_seq_01} does not depict when the actionservers are running it rather displays the time they are actively used	
	
	In the following subsections the execution and the procedure depicted in figure \ref{grocery_seq_01} and \ref{grocery_seq_02} will be explained in detail and include decisions made which let to the excact plan. Some more detail challanges will be explained aswell.

	% Knowledge: set-tables-source 
	At first the definition of the task is set in knowledge: We want to find objects on all the tables, so all the table surfaces are set as \texttt{source} and we want to place the found objects in shelves, so all the shelves are set as \texttt{target}. This procedure enables Knowledge, to generically work over \texttt{source} and \texttt{target} surfaces regardless of what they actually are.
	
	\subsection{Scan Table}
	
	% Manipulation: take pose action
	
	% NLP: Talk Request
	
	% Knowledge: get table-poses
Knowledge finds all the tables available, sorts them by their distance to the robot and then returns the list of their position to Planning, so Planning can determine where to go next.
	
	% Navigation: moveBaseAction
	
	% NLP
	
	% Manipuilation: TakePoseAction
	
	% Perception: Percieve and return data
	
	% Knowledge: Store Data
	
	% NLP: Talk Request
	
	% Manipulation: Take pose
	
	% 2x NLP: Talk
	
	% Knowledge: prolog_table_pose (???)
	
	% Navigation: MoveBaseAction
	
	\subsection{Grasp Object}
	
	% Knowledge: next_object
	
	% 2x NLP: Talk Request
	
	% Manipulation: GraspAction
	
	% Planning: This whole thing is gonna be looped 
	
	% NLP
	
	% Knowledge: Shelf Positions
Knowledge finds all the shelves available and sorts them, just like the tables before, by their distance to the robot. This way, the Robot can navigate to all the shelves starting with the nearest one to scan the objects already in place.
	
	% Navigation: Move to Shelf
	
	\subsection{Scan shelf floors}
	
	% Planning: Loop through all shelf floors
	
	% NLP
	
	% Manipulation: go into percieve pose
	
	% Perception: Percieve shelf
	
	% Knowledge: Store new Objects
	
	\subsection{Place Object}
	
	% Planning: Loop this while some items are left
	
	% Knowledge: shelf positions
	
	% navigation: move to shelf (???)
	
	% Knowledge: Object goal pose
	
	% Manipulation: TakePoseAction
	
	% NLP
	
	% Manipulation: TakePoseAction
	
	% NLP
	
	% Knowledge: table pose
	
	% Knowledge: next_object
	
	% 2x NLP
	
	% Planning: Finish
	
	
	\section{Conclusion}

	We achieved the task blash blash blash ...
	
	
	\endgroup
	
\end{document}
